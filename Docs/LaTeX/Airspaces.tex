% 	Airspaces.tex
%
%	Use this file to write a documentation about
%
%%%%%%%%%%%%%%%%%%%%%%%%%%%%%%%%%%%%%%%%%%%%%%%%%%%%%%%%%%%%%%%%%%%%%%%%%%%%%%%
%%%%%%%%%               Airspaces        					
%%%%%%%%%%%%%%%%%%%%%%%%%%%%%%%%%%%%%%%%%%%%%%%%%%%%%%%%%%%%%%%%%%%%%%%%%%%%%%%
%
%	Author :
%		Mr. Nobody
%
%%%%%%%%%%%%%%%%%%%%%%%%%%%%%%%%%%%%%%%%%%%%%%%%%%%%%%%%%%%%%%%%%%%%%%%%%%%%%%%



\subsection{Airspaces}

\begin{description}

\item[Line spacing]\index{line spacing}
        The spacing between two lines should be larger than the spacing 
        between two words to guide the eyes of the reader.

\item[Line length]\index{line length}
        The length of a line -- or when using multicolumn layout of a 
        column -- should be about 60 characters. When lines get longer they 
        are more difficult to read and it is easier to go to the wrong line 
        after finishing the current one. Increasing the linespacing may help a 
        little.
        When lines get to short it is difficult to set them justified, and you 
        will get lots of hyphenated words.
        
\item[Page layout]\index{page layout}
        Normal text pages should look the same throughout the document. 
        Figures, tables and special pages like the index need not appear in 
        the same layout but should take as much space as needed.
        
\item[Margin notes]\index{margin notes}
        Margin notes are often more suitable than footnotes because they 
        appear right next to the text they refer to. Special margin notes are 
        the ``attention sign'' or the ``dangerous bend'' that guide the user 
        to important parts of the text.
        
\item[Headings and Footings]\index{headings}\index{footings}
        Headings and footings should make it easier for the reader to orient
        himself in the document. If you expect readers to copy single pages
        from the document they should contain information about the paper as
        a whole, just in case you need more information or want to cite the
        whole paper.
        
        If you expect the document to change often (like software manuals),
        each page should contain a version information or at least a date.
        
\end{description}


\endinput
